\documentclass[aps,12pt]{revtex4}
\usepackage[a4paper]{geometry}
\usepackage{graphicx}
\usepackage{amssymb,amsfonts,amsmath,amsthm}
\usepackage{bm}
\usepackage{pslatex}
\usepackage{chemarr}
\usepackage{mathptmx}
\usepackage{bookman}


\begin{document}

\section{Regularizing}
We have $f_0$ and $f_1$ with opposite signs.
We compute $f_{1/2}$.
We define $h(u) = f(u) e^{\alpha u}$ such that
\begin{equation}
\begin{array}{lcl}
h_0  & = & f_0  \\
h_1  & = & f_1 e^{\alpha} \\
h_{1/2} & = & f_{1/2} e^{\alpha/2} = \dfrac{1}{2}( h_0+h_1 )\\
\end{array}
\end{equation}
 
\begin{equation}
	\dfrac{1}{2}( f_0+f_1 e^\alpha ) = f_{1/2} e^{\alpha/2} 
\end{equation} 

\begin{equation}
	f_0 - 2 f_{1/2} e^{\alpha/2}  + f_1 e^\alpha = 0
\end{equation} 

$$
	 Q = e^{\alpha/2} \text{~~~is a positive solution of~~~} f_0 - 2 f_{1/2} Q  + f_1 Q^2 = 0
$$

\begin{equation}
	\Delta' = f_{1/2}^2 - (f_0f_1) > 0, \;\; \vert \Delta' \vert > \vert f_{1/2} \vert
\end{equation}

\begin{equation}
	Q_\pm = \dfrac{ f_{1/2} \pm \sqrt{\Delta'} }{f_1} = \dfrac{f_0}{f_{1/2} \mp \sqrt{\Delta'}}
\end{equation}
There are always two solutions with opposite signs.
 
\begin{equation}
	Q_\pm^2 = \dfrac{2f_{1/2}Q_\pm - f_0}{f_1}
\end{equation}
 
\begin{equation}
	f_1 Q_\pm^2 =  2f_{1/2}Q_\pm - f_0
\end{equation}

 
\section{Opposite $f_0$ and $f_{1/2}$ $\implies$ same sign for $f_1$ and $f_{1/2}$}
 
 $$
 	h(u) \approx h_0 + 2 u (h_{1/2}-h_0) \implies u = \dfrac{1}{2} \dfrac{h_0}{h_0-h_{1/2}} \in ]0:\frac{1}{2}[
 $$
  
 $$
 	u =  \frac{1}{2}\dfrac{f_0}{f_0 - f_{1/2}Q}
 $$
 
 \begin{itemize}
 \item 
 $$
 	f_0>0, f_{1/2}<0 \implies Q = \frac{f_0}{f_{1/2}+\sqrt{\Delta'}} 
	\implies u = \dfrac{1}{2} \dfrac{1}{ 1 - \dfrac{f_{1/2}}{f_{1/2}+\sqrt{\Delta'}} } = \dfrac{1}{2} \dfrac{f_{1/2}+\sqrt{\Delta'}}{\sqrt{\Delta'}}
$$


\item 
 $$
 	f_0<0, f_{1/2}>0 \implies Q = \frac{f_0}{f_{1/2}-\sqrt{\Delta'}} 
	\implies u = \dfrac{1}{2} \dfrac{1}{ 1 - \dfrac{f_{1/2}}{f_{1/2}-\sqrt{\Delta'}} } = \dfrac{1}{2}  \dfrac{\sqrt{\Delta'}-f_{1/2}}{\sqrt{\Delta'}}
$$
 \end{itemize}
 
 \section{Same sign for $f_0$ and $f_{1/2}$ $\implies$ opposite sign for $f_1$ and $f_{1/2}$}

$$
	h(u) \approx h_{1/2} + 2(u-\dfrac{1}{2})(h_{1/2}-h_1) \implies u = \dfrac{1}{2}\dfrac{h_1 - 2h_{1/2}}{h_1-h_{1/2}}
	= \frac{1}{2} + \frac{1}{2} \dfrac{h_{1/2}}{h_{1/2}-h_1} \in ]\frac{1}{2}:1[
$$
 
  
  
\end{document}