\documentclass[aps,12pt]{revtex4}
\usepackage[a4paper]{geometry}
\usepackage{graphicx}
\usepackage{amssymb,amsfonts,amsmath,amsthm}
\usepackage{bm}
\usepackage{pslatex}
\usepackage{chemarr}
\usepackage{mathptmx}
\usepackage{bookman}


\begin{document}

$a$ is almost equal to $b$ iff
\begin{equation}
	\dfrac{\vert a - b \vert}{ \max(\theta,\min(\vert a \vert,\vert b \vert)) } \leq \epsilon
\end{equation}
with a small $\theta$.
When $b=0$, 
\begin{equation}
	\dfrac{\vert a \vert }{ \max(\theta,\vert a \vert) } \leq \epsilon
\end{equation}

The bisection methods divides by two the size of the interval at each iteration.
The relative size of the interval after $n$ iterations is
$$
	 2^{-n}
$$
Accordingly, 
$$
	n \simeq -\ln_2 \epsilon
$$

\end{document}