\documentclass[aps,12pt]{revtex4}
\usepackage[a4paper]{geometry}
\usepackage{graphicx}
\usepackage{amssymb,amsfonts,amsmath,amsthm}
\usepackage{bm}
\usepackage{pslatex}
\usepackage{chemarr}
\usepackage{mathptmx}
\usepackage{bookman}


\begin{document}

\section{Conservation}

Let $\vec \alpha$ be the representation of an unbalanced constraint.

\begin{equation}
	\langle \vec \alpha \vert \vec A \rangle = - x_s < 0
\end{equation}
We want to find the minimal $d \vec A$ such that:
\begin{equation}
	\langle \vec \alpha \vert \vec A + d\vec A \rangle = 0 \implies \langle \vec \alpha \vert d \vec A \rangle = x_s \implies d \vec A = \dfrac{x_s}{\vec\alpha^2} \vec \alpha
\end{equation}
The increase in concentration is:
\begin{equation}
	d \vec A^2 = \dfrac{x_s^2}{\vec \alpha^2} = \left( \dfrac{x_s}{\vert\vec\alpha\vert}\right) ^2
\end{equation}

\begin{equation}
	\vec A ' = \vec A + d \vec A = \vec{A} - \dfrac{1}{\vec \alpha^2} \langle \vec \alpha \vert \vec A \rangle \vec \alpha 
	= \dfrac{1}{\vec \alpha^2} \left[ \vec \alpha^2 \bm I - \vert \vec \alpha \rangle \langle \vec \alpha \vert \right] \vec A
\end{equation}


%Let $\vec \alpha_i$ be an unbalanced constraint so that
%\begin{equation}
%	\langle \vec \alpha_i \vert \vec A \rangle = - x_i < 0
%\end{equation}
%Let us assume that we have $n$ unbalanced constraints forming:
%\begin{equation}
%	\bm \alpha \vec A = - \vec x
%\end{equation}
%We want the minimal $d \vec A$ such that:
%\begin{equation}
%	\bm \alpha \vert \vec A + d \vec A \rangle = \vec 0 \implies \bm \alpha d\vec A = \vec x
%\end{equation}
%
%\begin{equation}
%	d \vec A = \bm \alpha ^ T \vec \lambda \implies d \vec A = \bm \alpha^T \left(\bm \alpha \bm \alpha ^T \right) ^{-1}\vec x 
%\end{equation}
%
%\begin{equation}
%	\vec{A} ' = \left[\bm I_n - \alpha^T \left(\bm \alpha \bm \alpha ^T \right)^{-1} \bm \alpha \right] \vec{A} 
%	= \dfrac{1}{\det \left(\bm \alpha \bm \alpha ^T\right) } \left[\det \left(\bm \alpha \bm \alpha ^T\right)\bm I_n - \alpha^T \left(\bm \alpha \bm \alpha ^T \right)^\ast \bm \alpha \right] \vec{A}
%\end{equation}
 	
\section{Solver}

Once the active equilibria are selected, each one has a computable 
affinity:
\begin{equation}
\label{eq:affinity}
	\mathcal A_i = \ln K_i - \ln Q_i = \ln K_i -  \sum_{i} \nu_{ij} \ln [A_j] %Q=prod/reac
\end{equation}
and
\begin{equation}
	d \mathcal A_i = - \sum_i \dfrac{\nu_{ij}}{[A_j]} \, d[A_j] \implies
	 \vec \nabla \mathcal A_i = 
	\begin{bmatrix}
	\vdots\\
	\frac{\nu_{ij}}{[A_j]}\\
	\vdots\\
	\end{bmatrix}
\end{equation}

The objective function to reduce is the RMS of affinities:
\begin{equation}
	W = \sqrt{ \dfrac{\sum_i \mathcal A_i^2}{N} } 
\end{equation}

\begin{equation}
	dW = \dfrac{1}{N} \dfrac{\sum_i \mathcal A_i \, d \mathcal A_i}{W}
	\implies
	\vec \nabla W = \dfrac{1}{N\cdot W} \left( \sum_i \mathcal A_i \vec \nabla \mathcal A_i \right)
\end{equation}

\begin{equation}
\partial_{\xi_i} W = \dfrac{1}{N\cdot W} \langle \left( \sum_k \mathcal A_k \vec \nabla \mathcal A_k \right) \vert \vec \nu_i \rangle
\end{equation}


\end{document}